\documentclass[11pt]{scrartcl}
%%
%
% This is a poster template with latex macros and using
% the University of Florida Logo.  For further information
% on making postscript, resizeing, and printing the poster file
% please see web site 
% http://www.phys.ufl.edu/~pjh/posters/poster_howto.html
% 
% N.B. This format is cribbed from one obtained from the University
% of Karlsruhe, so some macro names and parameters are in German
% Here is a short glosary:
% Breite: width
% Hoehe: height
% sPALTE: column
% Kasten: box
%
% All style files necessary are part of standard TeTeX distribution
% On the UF unix cluster you should not need to import these files
% specially, as they will be automatically located.  If you
% run on a local PC however, you will need to locate these files.
% At UF try /usr/local/TeTeX...
% 
% P. Hirschfeld 2/11/00
%
% The recommended procedure is to first generate a ``Special Format" size poster
% file, which is relatively easy to manipulate and view.  It can be
% resized later to A0 (900 x 1100 mm) full poster size, or A4 or Letter size
% as desired (see web site).  Note the large format printers currently
% in use at UF's OIR have max width of about 90cm or 3 ft., but the paper
% comes in rolls so the length is variable.  See below the specifications
% for width and height of various formats.  Default in the template is
% ``Special Format",  with 4 columns.
%%
%% 
%% Choose your poster size:
%% For printing you will later RESIZE your poster by a factor
%%        2*sqrt(2) = 2.828    (for A0)
%%        2         = 2.00     (for A1) 
%%  
%% 
\def\breite{400mm}     % Special Format. 
\def\hoehe{400mm}      % Scaled by 2.82 this gives 110cm x 90cm 

%\def\breite{390mm}     % Special Format. 
%\def\hoehe{319.2mm}      % Scaled by 2.82 this gives 110cm x 90cm 
\def\anzspalten{2}
%%
%%\def\breite{420mm}     % A3 LANDSCAPE
%%\def\hoehe{297mm}
%%\def\anzspalten{4}
%%
%% \def\breite{297mm}     % A3 PORTRAIT
%% \def\hoehe{420mm}
%% \def\anzspalten{3}
%%
%% \def\breite{210mm}     % A4 PORTRAIT
%% \def\hoehe{297mm}
%% \def\anzspalten{2}
%%
%%
%% Procedure:
%%   Generate poster.dvi with latex
%%   Check with Ghostview
%%   Make a .ps-file with ``dvips -o poster.ps poster''
%%   Scale it with poster_resize poster.ps S
%%   where S is scale factor
%%     for Special Format->A0 S= 2.828 (= 2^(3/2)))
%%     for Special Format->A1 S= 2 (= 2^(2/2)))
%% 
%% Sizes (European:)
%%   A3: 29.73 X 42.04 cm
%%   A1: 59.5 X 84.1 cm
%%   A0: 84.1 X 118.9 cm
%%   N.B. The recommended procedure is ``Special Format x 2.82"
%%   which gives 90cm x 110cm (not quite A0 dimensions).
%%
%% --------------------------------------------------------------------------
%%
%% Load the necessary packages
%% 
\usepackage{palatino}
\usepackage[latin1]{inputenc}
\usepackage{epsf}
\usepackage{graphicx,psfrag,color,pstricks,pst-grad}
\usepackage{amsmath,amssymb}
\usepackage{latexsym}
\usepackage{calc}
\usepackage{multicol}
\usepackage{url}
%%
%% Define the required numbers, lengths and boxes 
%%
\newsavebox{\dummybox}
\newsavebox{\spalten}
%\input psfig.sty

%%
%%
\newlength{\bgwidth}\newlength{\bgheight}
\setlength\bgheight{\hoehe} \addtolength\bgheight{-1mm}
\setlength\bgwidth{\breite} \addtolength\bgwidth{-1mm}

\newlength{\kastenwidth}

%% Set paper format
\setlength\paperheight{\hoehe}                                             
\setlength\paperwidth{\breite}
\special{papersize=\breite,\hoehe}

\topmargin -1in
\marginparsep0mm
\marginparwidth0mm
\headheight0mm
\headsep0mm


%% Minimal Margins to Make Correct Bounding Box
%\setlength{\oddsidemargin}{-2.44cm}
%\addtolength{\topmargin}{-3mm}
\setlength{\oddsidemargin}{-2.44cm}
\addtolength{\topmargin}{-3mm}
\textwidth\paperwidth
\textheight\paperheight

%%
%%
\parindent0cm
\parskip1.5ex plus0.5ex minus 0.5ex
\pagestyle{empty}




\definecolor{recoilcolor}{rgb}{1,0,0}
\definecolor{occolor}{rgb}{0,1,0}
\definecolor{pink}{rgb}{0,1,1}





\def\UberStil{\normalfont\sffamily\bfseries\large}
\def\UnterStil{\normalfont\sffamily\small}
\def\LabelStil{\normalfont\sffamily\tiny}
\def\LegStil{\normalfont\sffamily\tiny}

%%
%% Define some commands
%%
\definecolor{JG}{rgb}{0.1,0.9,0.3}

\newenvironment{kasten}{
  \begin{lrbox}{\dummybox}
    \begin{minipage}{\linewidth}}
    {\end{minipage}
  \end{lrbox}
  \raisebox{-\depth}{\psshadowbox[cornersize=absolute,linearc=14pt,framesep=1em]{\usebox{\dummybox}}}\\[0.5em]}
\newenvironment{spalte}{
  \setlength\kastenwidth{1.2\textwidth}
  \divide\kastenwidth by \anzspalten
  \begin{minipage}[t]{\kastenwidth}}{\end{minipage}}

%\renewcommand{\emph}[1]{{\color{red}\textbf{#1}}}


%\def\op#1{\hat{#1}}
\begin{document}
%%%%%%%%%%%%%%%%%%%%%%%%%%%%%%%%%%%%%%%%%%%%%%%%%%%%
%%%               Background                     %%%
%%%%%%%%%%%%%%%%%%%%%%%%%%%%%%%%%%%%%%%%%%%%%%%%%%%%
{\newrgbcolor{gradbegin}{0.5 0.5 1}
  \newrgbcolor{gradend}{1 1 1}%{1 1 0.5}%
  %\rput[cm](15.23,-15){
  \rput[tl]{0}(0,0){
  \includegraphics[width=\paperwidth]{logos/antena1} %lower resolution
  %\includegraphics[width=\paperwidth]{logos/alma_logo} %higher resolution
  }
\vfill}

%{\newrgbcolor{gradbegin}{0.5 0.5 1}%
%  \newrgbcolor{gradend}{1 1 1}%{1 1 0.5}%
%  \psframe[fillstyle=gradient,gradend=gradend,%
%  gradbegin=gradbegin,gradmidpoint=0.1](\bgwidth,-\bgheight)}
%\vfill
%%%%%%%%%%%%%%%%%%%%%%%%%%%%%%%%%%%%%%%%%%%%%%%%%%%%
%%%                     Header                   %%%
%%%%%%%%%%%%%%%%%%%%%%%%%%%%%%%%%%%%%%%%%%%%%%%%%%%%
\hfill
\psshadowbox{\makebox[0.95\textwidth]{%
%    \hfill
	\parbox[c]{5cm}{\includegraphics[width=7cm,height=!]{logos/chivo_logo}}
    \hfill
    \parbox[c]{0.5\linewidth}{%
      \begin{center}
		\textbf{\Huge {The ChiVO Library: Advanced Computational Methods for
Astronomy}}\\[0.5em]
		\textsc{\large Mauricio Araya$^{1}$, Mauricio Solar $^{1}$,
				Diego Mardones $^{2}$, Marcelo Mendoza $^{1}$, Camilo Valenzuela$^{1}$,
Teodoro Hochfarber$^{1}$, Mart\'in Villanueva$^{1}$, Marcelo Jara $^{1}$, Axel
Simonsen$^{1}$
		\\[0.3em]
		{ $^1$Universidad T\'ecnica Federico Santa Mar\'ia, Valpara\'iso, Chile}\\
		{ $^2$}Universidad de Chile, Santiago, Chile}
      \end{center}}
	\hfill
    \parbox[c]{2cm}{\includegraphics[width=4cm,height=!]{logos/usm_logo}}
	\hfill
\hfill}}\hfill\mbox{}\\

\hfill
\psshadowbox{\makebox[0.83\textwidth]{%
    \hfill
    \parbox[t]{0.8\linewidth}{%
\hfill{\large\bf{\color{red} ABSTRACT}}\hfill\mbox{}\\
The main objective of the Advanced Computational Astronomy Library (ACALib) is
to ensemble a coherent software package with the research on computational
methods for astronomy performed by the first phase of the Chilean Virtual
Observatory between years 2013 and 2015. During this period, researchers and
students developed functional prototypes, implementing state of the art
computational methods and proposing new algorithms and techniques. This
research was mainly focused on spectroscopic data cubes, as they strongly
require computational methods to reduce, visualize and infer astrophysical
quantities from them, and because most of the techniques are directly applicable
either to images or to spectra.
%ACALib is a software package that implements several algorithms and
%tools for analyzing spectroscopic data cubes, images and spectra. The library
%consist in a coherent framework for developing novel webservices for
%processing data on-line in the Chilean Virtual Observatory (ChiVO), but it also
%offers a generic API for developing stand-alone applications. The algorithms
%automatically connects to VO services, and the library is compatible with the
%SAMP protocol for interacting with other applications like Topcat or Aladin.

%The library is divided into 5 modules. The \emph{core} module has the main 
%classes that represent and manipulate astronomical data. The \emph{VO} module
%provides the workspace abstraction and VO communication interfaces.
%The \emph{synthetic} module generates synthetic data (spectral lines, flux
%distributions, meta-data, etc). The \emph{process} module contains the
%algorithms developed so far by ChiVO. At last, the \emph{graphic} module 
%will have the widgets and tools for 3D visualization of 
%spectroscopic cubes, clumps, surfaces, spectra, etc. Through the poster,
%the boxes in yellow are those components that are already implemented and
%integrated, the green ones are the ones that are already prototyped, but not
%integrated, and in blue/cyan are those that are currently under development.

%The library is strongly based on Astropy~\cite{} and uses vectorized
%computations through numpy and advanced algorithms from scipy and scikit-learn.
}
\hfill}}\hfill\mbox{}

\begin{lrbox}{\spalten}
  \parbox[t][0.71\textheight]{1.3\textwidth}{
   % \vspace*{0.2cm}
%%%%%%%%%%%%%%%%%%%%%%%%%%%%%%%%%%%%%%%%%%%%%%%%%%%%
%%%                 first column                 %%%             
%%%%%%%%%%%%%%%%%%%%%%%%%%%%%%%%%%%%%%%%%%%%%%%%%%%%
\begin{spalte}
     
	\begin{kasten}
        \section*{\hspace{0.2cm} {\color{red} ACALIB} }

      \begin{minipage}{0.6\textwidth}
                        %\strut\vskip -\baselineskip
                        %\hfill
         \begin{center}
                           \includegraphics[width=0.8\textwidth]{img/acageneral}
         \end{center}
\textbf{Color notation}:
\begin{itemize}
\item \textbf{yellow}: element already implemented and integrated
\item \textbf{green}: element  already prototyped, but not integrated
\item \textbf{blue/cyan}: element under development
\end{itemize}
                        %\vspace{0.1cm}
          \end{minipage}
		\begin{minipage}{0.4\textwidth}

\textbf{Key points}:
\begin{itemize}
\item Library that implements several algorithms and
tools for analyzing spectroscopic data cubes, images and spectra.
\item Coherent framework for developing novel \emph{webservices} for ChiVO
\item But also has an API for stand-alone applications (\texttt{python} interface)
\item Strongly grounded in \texttt{astropy} and \texttt{numpy}
\item Reuse some algorithms from \texttt{scipy}, \texttt{scikit-learn} and
\texttt{astroML}.
\item Algorithms automatically search into VO services 
\item Compatible with SAMP (e.g., connects with Topcat and Aladin)
\end{itemize}
\textbf{Modules}:
\begin{itemize}
\item \textbf{core}: main classes to manipulate astronomical data
\item \textbf{vo}: workspace abstraction and VO comm interfaces
\item \textbf{synthetic}: generates synthetic spectroscopic cubes
\item \textbf{process}: algorithms developed so far by ChiVO
\item \textbf{graphic}: widgets and tools for 3D visualization  
\end{itemize}

		\end{minipage} 
\mbox{}\\
      
\end{kasten}\hfill

	\begin{kasten}
        \section*{\hspace{0.2cm} {\color{red} CORE MODULE} }
			%\vspace{-0.1cm}
			\begin{minipage}{0.5\linewidth}
\textbf{Members}
\begin{itemize}
\item \textbf{AData} is an extension of astropy NDData
\begin{itemize}
\item Vectorized masked arrays: fast computations with missing values 
\item Metadata and WCS support: from astropy
\item Self-operations: rotate, scale, slice, stack, statistics, search, etc.
\item Transactions: deferred and online WCS/Meta consistency
\end{itemize}
\item \textbf{ATable} is an extension of astropy Table
\begin{itemize}
\item Interface: simpler than its parent
\item Metadata support: from astropy
\item Self-operations: statistics
\end{itemize}
\item \textbf{AContainer} is composed by a list of AData and ATable objects
\begin{itemize}
\item Namespace: it works as a namespace for astronomical data
\item Self-operations: Load and save FITS files
\end{itemize}
\end{itemize}
\textbf{Planned features} 
\begin{itemize}
\item Cython integration for less memory consuming
operations
\item MPI integration for multiprocessing environments
\item Hierarchical containers for supporting HDF5 format
\end{itemize}
			\end{minipage}
			\begin{minipage}{0.5\textwidth}
                        %\strut\vskip -\baselineskip
                        %\hfill
			\begin{center}
                        	\includegraphics[width=0.8\textwidth]{img/acacore}
			\end{center}
                        %\vspace{0.1cm}
          \end{minipage}

\mbox{}\\
          
	 \end{kasten}
    \end{spalte}
	 \hfill
%%%%%%%%%%%%%%%%%%%%%%%%%%%%%%%%%%%%%%%%%%%%%%%%%%%%
%%%               second column                  %%%             
%%%%%%%%%%%%%%%%%%%%%%%%%%%%%%%%%%%%%%%%%%%%%%%%%%%%
    \begin{spalte}
  

	\begin{kasten}
        \section*{\hspace{0.2cm} {\color{red} PROCESS MODULE} }
			%\vspace{-0.1cm}
			\begin{minipage}{0.5\linewidth}
\textbf{Algorithms} (OA = Original Algorithms)
\begin{itemize}
\item \textbf{Clumping}: detect clumps in an AData. Similar to CUPID package [1].
\begin{itemize}
\item GaussClumps: Mixture of Gaussians fitting \\ (Stutzki \& Gusten, 1990, ApJ) 
\item FellWalker: Agregation of hill-climbing paths \\ (Berry, 2015, A\&C)
\item BubbleClumps: Clustering of small Gaussians \\ (in preparation) (OA)
\end{itemize}
\item \textbf{ROI Detection}: index multi-resolution regions of interest in an AData.
\begin{itemize}
\item StructDetect: Morphological processing \\ (Mendoza et al., 2015, A\&C) [2] (OA)
\item WaveletDetect: Multi-scale detection in Wavelets space \\ (Gregorio et
al., 2015, SPIE) (OA)
\end{itemize}
\item DISPLAY: Learn dictionaries for line detection \\(Riveros, 2015, Thesis) (OA)
\item SpeLAR: Compute association rules for spectral lines \\ (Miranda, 2015,
Thesis) (OA)
\item Stacking: Automatic stacking of images \\ (Jara, 2015, Thesis) (OA)
\end{itemize}
\textbf{Under development}
\begin{itemize}
\item AutoDenoise: Denoising of images using deep autoencoders
\item LongVarStar: Detection of long-term variable stars using ML
\end{itemize}
			\end{minipage}
			\begin{minipage}{0.5\textwidth}
                        %\strut\vskip -\baselineskip
                        %\hfill
			\begin{center}
                        	\includegraphics[width=0.8\textwidth]{img/acaprocess}
			\end{center}
                        %\vspace{0.1cm}
          \end{minipage}
	 \end{kasten}

	\begin{kasten}

        \section*{\hspace{0.2cm} {\color{red} THE OTHER MODULES} }

         \begin{minipage}{0.25\textwidth}
                        %\strut\vskip -\baselineskip
                        %\hfill
         \begin{center}
                           \includegraphics[width=0.7\textwidth]{img/acavo}\\
  (VO)
         \end{center}
                        %\vspace{0.1cm}
          \end{minipage}
         \begin{minipage}{0.75\textwidth}
\begin{itemize}
\item \textbf{vo} : the workspace can host elements of the core, send them
through SAMP and obtain/export data from/to the VO. 
\item \textbf{synthetic} : module is an integrated version of ASYDO [3], it
can generate data for testing, training and validation.
\item \textbf{graphic}: this is a key module that we are implementing to validate 
the results of each algorithm.
\end{itemize}
                        %\strut\vskip -\baselineskip
                        %\hfill
         \begin{center}
                           \includegraphics[width=0.8\textwidth]{img/asydo}\\
(SYNTHETIC)
         \end{center}
                        %\vspace{0.1cm}
          \end{minipage}
	 \end{kasten}
    \end{spalte}
	 \hfill
     \hfill\mbox{}
}




\end{lrbox}
\hfill
\resizebox*{0.95\textwidth}{!}{
  \usebox{\spalten}}\hfill\mbox{}


%\hfill
%\resizebox*{0.95\textwidth}{!}{
%  \usebox{\spalten}}\hfill\mbox{}
%\vspace{1.0cm}
%\hspace{1cm}
%\psshadowbox[cornersize=absolute,linearc=14pt]{\makebox[0.923\textwidth]{%
% \hfill
% \parbox[t]{0.9\linewidth}{
%\begin{minipage}[t]{0.95\linewidth}
%\section*{ {\normalsize \color{red} Conclusions}}
%{%\footnotesize
%			\begin{minipage}[t]{0.50\linewidth}
%Despite the amount of data generated by observatories in Chile, there was no VO
%a year ago, and particularly, ALMA at the moment does not have services
%compatible with VO. The implementation of the ChiVO caused a series of
%complications that arise from the understanding of the needs of the
%astronomers, and how these are translated to data systems and services, using
%IVOA international regulations.
%
%The approach of the current development is focused on incremental prototypes,
%thus generating frequent deliverables to the system users (Astronomers) who are
%pleased by the progress made. However, there are always things to implement, or
%new ideas coming up. So far it has been possible to capture the astronomers
%requirements, and to address the study of the protocols and standards, 
%thus creating the data access services (SCS, SIA, SSA, TAP); 
%			\end{minipage} \hspace{0.2cm}
%			\begin{minipage}[t]{0.50\linewidth}
%these have access to the data base that use a relational data model (ObsCore),
%subjected to their own ChiVO architecture, which enables the interoperability
%of the services.
%
%Currently, the prototype is in its first delivery, and by mid-2014 it will be
%launched its second iteration, which will include the public data of the cycle
%0 of ALMA. Hence, more users will be familiarized with the system, having the
%chance to test in situ the current functionalities and limitations of the
%implementations. For the following versions it is considered the
%implementation of other IVOA architecture protocols, as the section of Registry
%and standards of access to resources, as VOSPace [1]. 
%Regarding the models and multidimensional data, as the ALMA cubes, it will be
%necessary to work in the creation or adaptation of the IVOA standards.
%			\end{minipage}
%}
%\end{minipage}
%}\hfill
%}}\hfill
\vfill
\mbox{}\\

\vfill

\$mbox{}

\hfill
\psshadowbox{\makebox[0.95\linewidth]{
\hfill
\begin{minipage}[t]{0.30\linewidth}
	{\scriptsize
	{\small\bf Acknowledgements}
	
	\renewcommand{\baselinestretch}{0.4}
	This work was partially financed by CONICYT through the FONDEF D11I1060 project and the ICHAA
79130008 project.
	}
	
\end{minipage}\hfill

\begin{minipage}[t]{0.60\linewidth}
	{\scriptsize
	\renewcommand{\baselinestretch}{1.0}
	{\small\bf References}

        [1] Berry et al. \emph{CUPID: Clump Identification and Analysis
Package}. ADASS 2013
        \renewcommand{\baselinestretch}{0.5}

	[2] Mendoza et al. \emph{Indexing data cubes for content-based searches in
radio astronomy}. Astronomy \& Computing 2015 (to appear)
	\renewcommand{\baselinestretch}{0.5}

	[3] Araya et al. \emph{Exorcising the Ghost in the Machine: Synthetic
Spectral Data Cubes for Assessing Big Data Algorithms}. ADASS 2014
	\renewcommand{\baselinestretch}{0.5}
	}
\end{minipage}
}}\hfill\mbox{}

\vfill

\begin{center}
\psshadowbox{\makebox[0.95\linewidth]{
\hfill
\begin{minipage}[t]{0.15\linewidth}
	\begin{tabular}{ccc}
	\includegraphics[width=!,height=1.5cm]{logos/alma} &
	\includegraphics[width=!,height=1.5cm]{logos/reuna} &
	\includegraphics[width=!,height=1.5cm]{logos/fondef}
	\end{tabular}
\end{minipage} \hfill
\begin{minipage}[t]{0.50\linewidth}
	\begin{center}
	\Huge{Chilean Virtual Observatory}
	\end{center}
\end{minipage} \hfill
\begin{minipage}[t]{0.15\linewidth}
	\begin{tabular}{ccc}
	\includegraphics[width=!,height=1.5cm]{logos/utfsm} &
	\includegraphics[width=!,height=1.5cm]{logos/uchile} &
	\includegraphics[width=!,height=1.5cm]{logos/puc}
	\end{tabular}
\end{minipage}
\hfill
}}
\end{center}



\vfill
\hfill
{\scriptsize
Contact e-mail: {\color{blue}maray@inf.utfsm.cl} / For more information about
Chilean Virtual Observatory, please visit our web site
{\color{blue}http://www.chivo.cl}
}
\hfill
\

\end{document}
